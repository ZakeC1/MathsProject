Based upon the five of the nine total languages shown in the previous chapter. Results achieved in the previous chapter demonstrated the similarities of languages within the same family, particularly French and German. Their uniqueness could also be seen when comparing the graphs of each language as they each varied. Especially with languages of a different family as the Indo-European were all clustered together more but the Japonic and Sino-Tibetan families were much more evenly distributed. 

Therefore the study of complex graphical properties such as trophic coherence, various centrality values and clustering have proved beneficially in distinguishing graphs from one another as well as a tool to represent possible information based upon the graph.  

We summarise each graph property and the correlations that were derived from them in the next section.

\section{Final Correlations}
Based on each graph property that was studied, we take their correlations within the story corpus used and summarised as follows.

\subsection{Local Clustering Coefficient}
Local clustering with the lexical graphs corresponded to words that would most likely be involved in triples (a 3-cycle). However because of the common reuse of words, there were usually no triples that represented anything unique to the language. A way to improve the usefulness of local clustering in lexical analysis would be to use use specific motifs that could prove to generate a better correlation. Currently a generic 3-cycle is used for the local clustering calculations so using specific motifs may generate interesting results..

\subsection{Betweenness Centrality}
Betweenness centrality was chosen as it identifies the vertices with most control of informational flow. In the context of language analysis, betweenness values corresponded to the most commonly used words in each language which is further proven by the Zipf curve as mentioned in the English analysis. This was further verified when compared with the frequency of the words/vertices for every language dataset.

\subsection{Closeness Centrality}
Closeness centrality determines the vertices that on average have the shortest connections to all the other vertices within the graph. For lexical analysis, closeness centrality correlated to the vertices whom are most likely to isolate the rest of it's sentence. This may be because the rest of the sentence uses more unique words which means that the vertex of high closeness is the closest to these words as well as the previous words whom did not have a high closeness. Therefore the reason these such vertices had a high closeness and it's correlation in lexical terms.

\subsection{Page Rank}
Page ranks finds the vertices who are seen as important by finding the backlinks that refer to the vertex and other vertices which reference these vertices. This is similar to betweenness as both identified the vertices that held the most importance within the text so the ranks were similar to the betweenness but has more reference to the predecessors of the vertices. This causes the slight different of the betweenness graph and the page rank graph.

\subsection{Trophic Level}
Finally tropic levels are useful in determining structure and flow in a graph. Within lexical analysis, the flow of the sentences were represented by the ranges of trophic levels which were normalised to 0-10 like with the other values. Consequently were used as the y-positioning downwards from 0-10 so that the sentence flow was represented correctly. Trophic levels for each vertex represented the average position that they belonged to within a sentence with 0 at the start and 10 at the end. Languages with better sentence structure had more evenly distributed levels due to the fact that their sentence length wasn't as varied to other languages. 

Additionally, through trophic levels visualisation, each graph property in the previous chapter had key subsegments of the y-range which held the relative graph properties high value. For example, in the betweenness graphs, the vertices of high betweenness value were centralised which means that they appeared most commonly in the middle of sentences. This is grammatically true because words of importance either appeared as a bridge in a sentence like "and" or are used commonly throughout the sentence. Thus leading to an average closer to the centre. A similar trend was seen with local clustering where the vertices/words of high local clustering were lower in the y-range.
\\

\noindent In conclusion, each graph property have been researched for their use and benefits within a graph. The calculations for each complex graphical property was also given along with the possible variations which depended on the graph types. I.e. the weights, whether the graph is directed, self loops etc. After detailed descriptions of various properties were given, specific ones were transferred and used on text corpuses based on various different languages and their corresponding language family. This was possible through the use of programming in Python code to generate a dataset from a text input (the story corpus for each language). The dataset can then be used to formalise a graph so that the calculations for each complex graph property can be calculated from. Furthermore, the values of each property are normalised and applied onto the positions of each vertex to generate a new graph that provides a visual representation for the dataset. Each new graph represents a different complex graph property which achieves the aim of my research. Therefore the newly generated visualisations of the graph have proven to be useful in analysis and to highlight key components of interest by showing them as extremes of clusters within the new graph. Results found can then be expanded onto a larger dataset to predict the key areas of the larger text such as the words that would be of interest. This also means that when given another graph of the same language but with unknown vertices, the graph generated based on the corpus can be used to predict and identify what the unknown vertices could be. 

\section{Further works and Applications}
Briefly mentioned in the last part, the visualisations can be used as a prediction tool within the same language as similar vertices that represent each word would correspond to a similar location in this visualisation. As well as this expansion, the visualisation can be used to analyse new or unknown languages through the comparison of graphs since languages of the same family produce graphs that have resemblances. Once the best language family is chosen, predictions of what components of the graph represents can be given from the facts of other languages within the same language family. Improvements for each specific language representations can be remedied since languages such as Japanese and Chinese don't use words as their sentence structure is more complex than most other languages.

Therefore the current works on lexical analysis and complex graph properties can be taken further by introducing unknown languages, expanding on the graph properties mentioned such as using specific motifs for the local clustering or the application onto other subject areas rather than lexical analysis.

Other areas that could be analysed includes neuroscience. The brain \cite{de2014graph} can be regarded as a complex network of neurons where each vertex represents units or specific regions and the edges represents the links or connections between them. Analysis of neurological diseases such as schizophrenia, dementia and Alzheimer's can be used to generated relative graph visualisations. These visualisations can be studied further to find common links and correlations within each neurological disease so that action could be taken if early signs of similar correlations are found within a healthy brain. A similar idea was studied based on detecting changes in the brain which has Alzheimer's using graph theory \cite{10.1093/braincomms/fcaa129}.
In conclusion graph theory is crucial in representing various types of information in the real world from friendship groups to the complex structures of the brain. Many subject areas can be represented as a graph which means that they can undergo similar graphical analysis and visualisations which experimented on in Chapters 3 and 4 to deduce correlations and key components of interest in the relative graph.
