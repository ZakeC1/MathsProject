\chapter*{Abstract}
In this project, we study the background of graph theory and explain specific complex graph properties. These properties include trophic levels and coherence, the centrality values such as betweenness and closeness, the clustering coefficients and properties based on webpage traversal like HITs or page rank. Each are explored in detail with the methods of calculation based on various graph types. The graph type that we focus on is the directed graphs. This was chosen in consideration for linguistic analysis which I have undertaken in a later chapter. To analyse directed graphs based on the multiple formulas for each graph property, I have programmed a tool to generate graph visualisations. The tool uses the formulas for each graph property and calculates the relative value for each vertex in a graph. This can then be formalised into a dataset and used to generate graph visualisations. The vertex positions of the visualisations are amended based on the various graph properties. Positional changes is beneficial since it highlights key areas and certain correlations for the different languages that I have chosen to analyse. The languages in this project being English, German, French, Japanese, and Chinese. Detailed analysis of each language based on the graph properties is described in this project. Finally, the results are summarised with the future possibilities of graph analysis.